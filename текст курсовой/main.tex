\documentclass[a4paper,12pt]{article}
\usepackage[english,russian]{babel}   
\usepackage[utf8]{inputenc}
\usepackage{amssymb,amsfonts,amsmath,enumerate,titlesec,indentfirst,graphicx}
\usepackage[left=30mm, top=20mm, right=20mm, bottom=20mm, nohead, footskip=10mm]{geometry} % настройки полей документа
\usepackage{amsthm}

\graphicspath{{pictures/}}
\DeclareGraphicsExtensions{.png,.jpg}

\newcommand{\sectionbreak}{\clearpage}

\renewcommand{\emptyset}{\varnothing} %символ пустого множества
\renewcommand{\bar}[1]{\overline{#1}} %растягивание всевозможных подчеркиваний

%\theoremstyle{definition}
\newtheorem{theorem}{Теорема}[section]
\newtheorem{definition}{Определение}[section]
\newtheorem{proposal}{Утверждение}[section]
\newtheorem*{remark}{Замечание}

\renewcommand\qedsymbol{$\blacktriangleleft$} %символ QED

\parindent=1.25cm %отступ абзаца

\begin{document}
    \begin{titlepage}
    \begin{center}
        {\large МИНОБРНАУКИ РОССИИ
            
            ~
            
            Федеральное государственное бюджетное образовательное учреждение высшего образования
                
            «Ярославский государственный университет им. П.Г.Демидова»
            
            ~
            
            Кафедра компьютерной безопасности и
            математических методов обработки информации}
        \vfill
        
        {\large Курсовая работа}
        
        ~
        
        \textbf{{\large Алгоритм Тарского: описание и реализация}}
    \end{center}
    \vfill
    
    \newlength{\ML}
    \settowidth{\ML}{«\underline{\hspace{0.7cm}}» \underline{\hspace{2cm}}}
    \hfill\begin{minipage}{0.4\textwidth}
        Научный руководитель
        
        профессор, д-р ф.-м.н.
        \\
        \underline{\hspace{\ML}} В.\,Г.~Дурнев\\
        «\underline{\hspace{0.7cm}}» \underline{\hspace{2cm}} 2020 г.
    \end{minipage}%
    \bigskip
    
    \hfill\begin{minipage}{0.4\textwidth}
        Студент группы КБ-41СО
        \\
        \underline{\hspace{\ML}} Р.\,А.~Гибадулин\\
        «\underline{\hspace{0.7cm}}» \underline{\hspace{2cm}} 2020 г.
    \end{minipage}%
    \vfill
    
    \begin{center}
        Ярославль, 2020 г.
    \end{center}
\end{titlepage}

    \renewcommand*\contentsname{Оглавление}
    \tableofcontents
    
    \section*{Введение}
	\addcontentsline{toc}{section}{Введение}

Пусть $\mathcal{A}$~--- формула логики высказываний. Задача: определить, является ли формула $\mathcal{A}$ тождественно истинной. В некотором смысле это <<трудная>> задача, однако очень просто предложить алгоритм для ее решения, но который будет <<не эффективным>>~--- алгоритм Британского музея. Рассмотрим другую задачу: определить, является ли формула \textbf{логики предикатов} тождественно истинной. Для данной задачи алгоритм перебора в общем случае уже не применим, так как множество значений переменных не обязано быть конечным. Но оказывается, для некоторых языков логики предикатов существуют алгоритмы решающие эту задачу. Одним из таких алгоритмов и является алгоритм Тарского, описанию и реализации которого посвящена данная работа.

\textbf{Цели работы:} изучить и описать алгоритм Тарского, и реализовать его в виде компьютерной программы.

\textbf{Задачи:}
\begin{itemize}
	\item Ввести определения, сформулировать и доказать утверждения необходимые для описания алгоритма Тарского;
	\item Реализовать компьютерную программу, которая по формуле элементарной алгебры без параметров   введенной с клавиатуры, строит эквивалентную бескванторную формулу того же языка. 
\end{itemize}

В первой части данной работы будет определен язык элементарной алгебры, дано определение элиминации кванторов и сформулировано утверждения о ней. Далее пойдет речь об идеях, на которых основан алгоритм, будут определены таблицы Тарского. Затем будут даны определения полунасыщенной и насыщенной систем многочленов, после чего будет описан метод построения таблиц Тарского, что практически завершит описание алгоритма Тарского.

Во второй части подробно рассматривается программа, написанная и отлаженная автором работы, а именно описано как происходит распознавание формулы, какие при этом используются алгоритмы, как организованно представление формул, описываются реализации насыщения системы многочленов и построения таблицы Тарского. 





    \section{Алгоритм Тарского}

вступление

\subsection{Элементарная алгебра}

Неформальное описание элементарной алгебры. Смотри первоисточник и википедия.

\subsection{Язык элементарной алгебры}

Формальное описание языка: Ссылка на учебник Дурнева. Сигнатура языка и алгебраическая система. Обратить внимание на $\mathbb{R}$. Ввести понятие нуль-местного предиката. Примеры утверждений в этом языке.
Замечания по про: перенос в неравенствах, про дроби, про коэффициенты многочленов (что они из Q и вообще хватит только 0, 1, -1), про операции -- хватит + и *, про нульместный предикат, про то что хватит < и =, 

определение: простой формулой будем называть формулу, в которой любая подформула вида QxA не содержит свободных вхождений переменных. 
\begin{definition}\label{simpleFormula}
    Формула $\mathcal{A}$ языка элементарной алгебры называется \textbf{простой}, если все подформулы формулы $\mathcal{A}$ вида $(Qx)\mathcal{B}$, где $Q$ -- квантор, не содержат свободных вхождений переменных.
\end{definition}


\subsection{Поле действительных чисел}

Теоремы из матана и алгебры про многочлены (ссылки на соответствующие книги), нули монотонной функции, монотонность многочленов, расположение корней производной (картинка нужна), значение многочлена в точке равно значению остатка в точке
Система из нескольких многочленов. Производная произведения.

\subsection{Элиминация кванторов}

Элиминация кванторов определение. Рассуждение -- хотим проверять истинность формул. Умеем это делать в логике высказываний. Значит "избавившись от кванторов" мы сведем задачу к задаче проверки истинности утверждения в логике высказываний.
Пусть алгоритм элиминации существуют, тогда: общая схема индуктивного "алгоритма" для любого языка -- нет кванторов -- задача решена, один квантор -- применяем алгоритм элиминации, получаем бескванторную, несколько кванторов -- элиминируем "внутренний", получили n-1 квантор, задача решена.

1

2

3

4

5

6

7

8

9

Пусть алгоритм $A$ такой, что $A\left(\left(Qx\right)\mathcal{A}\right) = \mathcal{B}$, где $\mathcal{A}$ и $\mathcal{B}$ -- бескванторные формулы языка элементарной алгебры, и формулы $(Qx)\mathcal{A}$ и $\mathcal{B}$ эквивалентны. Тогда выполнено следующее утверждение:
\begin{proposal}
    Если алгоритм $A$ существует, то существует алгоритм $B$ такой, что для любой формулы $\mathcal{A}$ языка элементарной алгебры $B\left(\mathcal{A}\right)$ -- бескванторная формула, эквивалентная $\mathcal{A}$.
\end{proposal}   
\begin{proof}
    Определим алгоритм $B$ следующим образом:
    \begin{itemize}
        \item Если $\mathcal{A}$ -- бескванторная формула, то $B\left(\mathcal{A}\right) = \mathcal{A}$;
        \item Если $\mathcal{A} = \left(Qx\right)\mathcal{B}$, то $B\left(\mathcal{A}\right) = A\left(\left(Qx\right)B\left(\mathcal{B}\right)\right)$. Формула $B\left(\mathcal{B}\right)$ -- бескванторная по построению $B$, следовательно запись $A\left(\left(Qx\right)B\left(\mathcal{B}\right)\right)$ корректна, при этом $B\left(\mathcal{B}\right)$ эквивалентна $\mathcal{B}$, следовательно, $\left(Qx\right)\mathcal{B}$ эквивалентна $\left(Qx\right)B\left(\mathcal{B}\right)$, а значит $\mathcal{A}$ эквивалентна $B\left(\mathcal{A}\right)$. Также заметим, что длина формулы $\mathcal{B}$ строго меньше длины формулы $\mathcal{A}$;
        \item Если $\mathcal{A}$ не удовлетворяет предыдущим условиям, то
        \begin{itemize}
            \item либо $\mathcal{A} = \lnot \mathcal{B}$, тогда $B\left(\mathcal{A}\right) = \lnot B\left(\mathcal{B}\right)$,
            \item либо $\mathcal{A} = \mathcal{B} * \mathcal{C}$, тогда $B\left(\mathcal{A}\right) = B\left(\mathcal{B}\right) * B\left(\mathcal{C}\right)$, где $* \in \left\{\lor, \&, \to\right\}$.
        \end{itemize}
        При этом длины формул $\mathcal{B}$ и $\mathcal{C}$ меньше длины формулы $\mathcal{A}$.
    \end{itemize}
    Алгоритм $B$ определен рекурсивно, при этом на каждом этапе на вход $B$ подаётся формула меньшей длины, следовательно, алгоритм $B$ является конечным, и на каждом шаге выход алгоритма -- бескванторная эквивалентная формула.
\end{proof} 
\begin{remark}
    Данное утверждение верно и для других языков логики предикатов.
\end{remark}

Таким образом, для элиминации кванторов в произвольной формуле достаточно построить алгоритм $A$.

\subsection{«Алгоритм» Тарского}

Изначально предложенный Тарским <<решающий метод>> (decision method (ссылка на работу Тарского)) имел другой вид, но Тарский был первым. А алгоритм, который приведен в данной работе, был предложен уже <<последователями>> Тарского, однако в литературе (отсылка на Матиясевича) именно этот алгоритм носит имя Тарского.

Описываем алгоритм, в котором есть слова "ищем корни многочленов"
сформулировать список того, почему это не настоящий алгоритм.

\subsection{Алгоритм Тарского}

Сначала рассмотрим базовый случай

Напоминаем про производную произведения. Что такое таблица Тарского. Сокращенная таблица Тарского. Свойства Таблицы Тарского. Насыщенная система. Алгоритм построения таблицы Тарского. Окончательное описание алгоритма.

Теперь перейдем с общему случаю

Особенности общего случая. Над каким полем многочлены. Описание алгоритма: ветвление в листах базовый случай

Теорема Тарского (2 формулировки)

Попробовать доказать корректность алгоритма, воспользовавшись утверждением из википедии про упрощение формулы.

Оценки: количество многочленов, количество многочленов в насыщенной системе, размеры таблицы, количество ветвлений.

МИНИЗАКЛЮЧЕНИЕ: существование алгоритма считалось невозможным. Давайте реализуем его.






    \section{Программная реализация}

Для реализации программы, которая по простой формуле (определение дать) языка элементарной алгебры, вводимой пользователем с клавиатуры, строит эквивалентную бескванторную формулу, были выбраны язык C\#, платформа .NET Core 3.1 и спецификация .NET Standard 2.1 \cite{TroelsonNet}. Такой выбор обусловлен рядом причин:
\begin{itemize}
    \item Язык C\# -- это объектно-ориентированный язык программирования, а данная парадигма программирования позволяет абстрактно описывать объекты, в том числе и математические объекты; 
    \item .NET Core и .NET Standard -- это современные, развивающиеся и востребованные кроссплатформенные технологии с открытым исходным кодом;
    \item Личные предпочтения автора.
\end{itemize}
Для поэтапного создания программы, были сформулированы и решены следующие задачи:
\begin{itemize}
    \item Разработать библиотеку классов для объектов языка элементарной алгебры: символы алфавита, термы, формулы;
    \item Реализовать ввод логических формул;
    \item Разработать библиотеку классов для рациональных чисел и многочленов от одной вещественной переменной с рациональными коэффициентами;
    \item Реализовать алгоритм Тарского для базового случая;
    \item Реализовать алгоритм элиминации кванторов(ссылка на утверждение о сведении задачи к алгоритму тарского) для простых формул языка элементарной алгебры;
    \item Обеспечить минимальное покрытие юнит-тестами;
    \item Собрать все библиотеки и модули в единый программный комплекс.
\end{itemize}
Написание программы осуществлялось в среде разработки Microsoft Visual Studio 2019, доступной для использования в некоммерческих целях и скачивания на официальном сайте: https://visualstudio.microsoft.com/ru/.

Также была рассмотрена перспектива реализации алгоритма Тарского для общего случая. 

\subsection{Представление формул}
форма представления символов и формул языка

Ввести понятие нуль-местного предиката.

\subsection{Система ввода}
Ввод формул

\subsection{Реализация базового случая}
Часть математики, часть алгоритм. Runner. Покрытие тестами. Консольное приложение

SimpleTarskiAlgorithmRunner

\subsection{Юнит-тестирование}
тесты

\subsection{Результаты работы}
Демонстрация результатов

\subsection{Общий случай}
Часть математики. Описание нерешённых задач.




    \section*{Заключение}
	\addcontentsline{toc}{section}{Заключение}

	Таким образом, цель данной работы достигнута, все поставленные задачи решены. В работе описан языка элементарной алгебры, приведены примеры задач элементарной алгебры, достаточно подробно описан алгоритм элиминации кванторов в этом языка~--- алгоритм Тарского. По мимо этого, алгоритм был реализован в виде компьютерной программы для достаточно важного случая~--- для формул без параметров. Эта программа включает в себя систему ввода формул языка элементарной алгебры, библиотеки классов для представления объектов этого языка и собственно реализацию алгоритма Тарского.

	В заключении хочется отметить, что алгоритм Тарского далеко не самый эффективный алгоритм, но он был первым в своем роде. Именно сконструировав этот алгоритм Альфред Тарский доказал, что элементарная алгебра допускает элиминацию кванторов, хотя до этого многие годы это считалось невозможным. 
	
	Для систем компьютерного доказательства, которые в ближайшем будущем станут очень востребованными (формальная верификация компьютерных программ), алгоритм Тарского вряд ли применим из-за крайне высокой трудоемкости. Поэтому можно продолжать работу в данному направлении и изучать другие алгоритмы элиминации квантров. А с точки зрения математики, интересен вопрос, а какие ещё языки допускают элиминацию кванторов? Поэтому автор продолжит работу в данном направлении.

    \begin{thebibliography}{99}
    \bibitem{lect1}
    Алгоритм Тарского. Лекция 1 // Лекториум. URL: https://www.lektorium.tv/lecture/31079 (дата обращения: 01.12.2019).

    \bibitem{lect2}
    Алгоритм Тарского. Лекция 2 // Лекториум. URL: https://www.lektorium.tv/lecture/31080 (дата обращения: 01.12.2019).

    \bibitem{Gibadulin1}
    Гибадулин Р. А. Алгоритм поиска вывода в Исчислении Высказываний и его программная реализация // Современные проблемы математики и информатики : сборник научных трудов молодых ученых, аспирантов и студентов. / Яросл. гос. ун-т им. П. Г. Демидова. -- Ярославль : ЯрГУ, 2019. -- Вып. 19. -- С. 28-37.

    \bibitem{DurnevML}
    Дурнев, В. Г. Элементы теории множеств и математической логики: учеб. пособие / Яросл. гос. ун-т. им. П. Г. Демидова, Ярославль, 2009 -- 412 с.

    \bibitem{Matiyasevich}
    Матиясевич, Ю. В. Алгоритм Тарского // Компьютерные инструменты в образовании. -- 2008. -- № 6. -- С. 14.

    \bibitem{TroelsonNet}
    Троелсен, Э. Язык программирования С\# 7 и платформы .NET и .NET Core / Э. Троелсен, Ф. Джепикс; пер. с англ. и ред. Ю.Н. Артеменко. -- 8-е изд. -- М.; СПб.: Диалектика, 2020. -- 1328 с.

    \bibitem{Tarski}
    Tarski, A. A Decision Method for Elementary Algebra and Geometry: Prepared for Publication with the Assistance of J.C.C. McKinsey, Santa Monica, Calif.: RAND Corporation, R-109, 1951. 

\end{thebibliography}

\addcontentsline{toc}{section}{\refname}
    
    \section*{Приложение А}
	\addcontentsline{toc}{section}{Приложение А}

	Ссылка на репозиторий на GitHub~--- https://github.com/romarioGI/Coursework2k19-2k20	

\end{document}