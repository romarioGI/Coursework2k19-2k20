\documentclass[a4paper,12pt]{article}
\usepackage[english,russian]{babel}   
\usepackage[utf8]{inputenc}
\usepackage{amssymb,amsfonts,amsmath,enumerate,titlesec,indentfirst,graphicx}
\usepackage[left=30mm, top=20mm, right=20mm, bottom=20mm, nohead, footskip=10mm]{geometry} % настройки полей документа
\usepackage{amsthm}

\graphicspath{{pictures/}}
\DeclareGraphicsExtensions{.png,.jpg}

\newcommand{\sectionbreak}{\clearpage}

\renewcommand{\emptyset}{\varnothing} %символ пустого множества
\renewcommand{\bar}[1]{\overline{#1}} %растягивание всевозможных подчеркиваний

%\theoremstyle{definition}
\newtheorem{theorem}{Теорема}[section]
\newtheorem{definition}{Определение}[section]
\newtheorem{proposal}{Утверждение}[section]
\newtheorem*{remark}{Замечание}

\renewcommand\qedsymbol{$\blacktriangleleft$} %символ QED

\parindent=1.25cm %отступ абзаца

\begin{document}
    \input{титульный лист.tex}

    \renewcommand*\contentsname{Оглавление}
    \tableofcontents
    
    \section*{Введение}
	\addcontentsline{toc}{section}{Введение}

Введение	


    \section{Алгоритм Тарского}

вступление

\subsection{Элементарная алгебра}

Неформальное описание элементарной алгебры. Смотри первоисточник и википедия.

\subsection{Язык элементарной алгебры}

Формальное описание языка: Ссылка на учебник Дурнева. Сигнатура языка и алгебраическая система. Обратить внимание на $\mathbb{R}$. Ввести понятие нуль-местного предиката. Примеры утверждений в этом языке.
Замечания по про: перенос в неравенствах, про дроби, про коэффициенты многочленов (что они из Q и вообще хватит только 0, 1, -1), про операции -- хватит + и *, про нульместный предикат, про то что хватит < и =, 

определение: простой формулой будем называть формулу, в которой любая подформула вида QxA не содержит свободных вхождений переменных. 
\begin{definition}\label{simpleFormula}
    Формула $\mathcal{A}$ языка элементарной алгебры называется \textbf{простой}, если все подформулы формулы $\mathcal{A}$ вида $(Qx)\mathcal{B}$, где $Q$ -- квантор, не содержат свободных вхождений переменных.
\end{definition}


\subsection{Поле действительных чисел}

Теоремы из матана и алгебры про многочлены (ссылки на соответствующие книги), нули монотонной функции, монотонность многочленов, расположение корней производной (картинка нужна), значение многочлена в точке равно значению остатка в точке
Система из нескольких многочленов. Производная произведения.

\subsection{Элиминация кванторов}

Элиминация кванторов определение. Рассуждение -- хотим проверять истинность формул. Умеем это делать в логике высказываний. Значит "избавившись от кванторов" мы сведем задачу к задаче проверки истинности утверждения в логике высказываний.
Пусть алгоритм элиминации существуют, тогда: общая схема индуктивного "алгоритма" для любого языка -- нет кванторов -- задача решена, один квантор -- применяем алгоритм элиминации, получаем бескванторную, несколько кванторов -- элиминируем "внутренний", получили n-1 квантор, задача решена.

1

2

3

4

5

6

7

8

9

Пусть алгоритм $A$ такой, что $A\left(\left(Qx\right)\mathcal{A}\right) = \mathcal{B}$, где $\mathcal{A}$ и $\mathcal{B}$ -- бескванторные формулы языка элементарной алгебры, и формулы $(Qx)\mathcal{A}$ и $\mathcal{B}$ эквивалентны. Тогда выполнено следующее утверждение:
\begin{proposal}
    Если алгоритм $A$ существует, то существует алгоритм $B$ такой, что для любой формулы $\mathcal{A}$ языка элементарной алгебры $B\left(\mathcal{A}\right)$ -- бескванторная формула, эквивалентная $\mathcal{A}$.
\end{proposal}   
\begin{proof}
    Определим алгоритм $B$ следующим образом:
    \begin{itemize}
        \item Если $\mathcal{A}$ -- бескванторная формула, то $B\left(\mathcal{A}\right) = \mathcal{A}$;
        \item Если $\mathcal{A} = \left(Qx\right)\mathcal{B}$, то $B\left(\mathcal{A}\right) = A\left(\left(Qx\right)B\left(\mathcal{B}\right)\right)$. Формула $B\left(\mathcal{B}\right)$ -- бескванторная по построению $B$, следовательно запись $A\left(\left(Qx\right)B\left(\mathcal{B}\right)\right)$ корректна, при этом $B\left(\mathcal{B}\right)$ эквивалентна $\mathcal{B}$, следовательно, $\left(Qx\right)\mathcal{B}$ эквивалентна $\left(Qx\right)B\left(\mathcal{B}\right)$, а значит $\mathcal{A}$ эквивалентна $B\left(\mathcal{A}\right)$. Также заметим, что длина формулы $\mathcal{B}$ строго меньше длины формулы $\mathcal{A}$;
        \item Если $\mathcal{A}$ не удовлетворяет предыдущим условиям, то
        \begin{itemize}
            \item либо $\mathcal{A} = \lnot \mathcal{B}$, тогда $B\left(\mathcal{A}\right) = \lnot B\left(\mathcal{B}\right)$,
            \item либо $\mathcal{A} = \mathcal{B} * \mathcal{C}$, тогда $B\left(\mathcal{A}\right) = B\left(\mathcal{B}\right) * B\left(\mathcal{C}\right)$, где $* \in \left\{\lor, \&, \to\right\}$.
        \end{itemize}
        При этом длины формул $\mathcal{B}$ и $\mathcal{C}$ меньше длины формулы $\mathcal{A}$.
    \end{itemize}
    Алгоритм $B$ определен рекурсивно, при этом на каждом этапе на вход $B$ подаётся формула меньшей длины, следовательно, алгоритм $B$ является конечным, и на каждом шаге выход алгоритма -- бескванторная эквивалентная формула.
\end{proof} 
\begin{remark}
    Данное утверждение верно и для других языков логики предикатов.
\end{remark}

Таким образом, для элиминации кванторов в произвольной формуле достаточно построить алгоритм $A$.

\subsection{«Алгоритм» Тарского}

Изначально предложенный Тарским <<решающий метод>> (decision method (ссылка на работу Тарского)) имел другой вид, но Тарский был первым. А алгоритм, который приведен в данной работе, был предложен уже <<последователями>> Тарского, однако в литературе (отсылка на Матиясевича) именно этот алгоритм носит имя Тарского.

Описываем алгоритм, в котором есть слова "ищем корни многочленов"
сформулировать список того, почему это не настоящий алгоритм.

\subsection{Алгоритм Тарского}

Сначала рассмотрим базовый случай

Напоминаем про производную произведения. Что такое таблица Тарского. Сокращенная таблица Тарского. Свойства Таблицы Тарского. Насыщенная система. Алгоритм построения таблицы Тарского. Окончательное описание алгоритма.

Теперь перейдем с общему случаю

Особенности общего случая. Над каким полем многочлены. Описание алгоритма: ветвление в листах базовый случай

Теорема Тарского (2 формулировки)

Попробовать доказать корректность алгоритма, воспользовавшись утверждением из википедии про упрощение формулы.

Оценки: количество многочленов, количество многочленов в насыщенной системе, размеры таблицы, количество ветвлений.

МИНИЗАКЛЮЧЕНИЕ: существование алгоритма считалось невозможным. Давайте реализуем его.






    \section{Программная реализация}

Для реализации программы, которая по простой формуле (определение \ref{simpleFormula}) языка элементарной алгебры, вводимой пользователем с клавиатуры, строит эквивалентную бескванторную формулу, были выбраны язык C\#, платформа .NET Core 3.1 и спецификация .NET Standard 2.1 \cite{TroelsonNet}. Такой выбор обусловлен рядом причин:
\begin{itemize}
    \item Язык C\# -- это объектно-ориентированный язык программирования, а данная парадигма программирования позволяет абстрактно описывать объекты, в том числе и математические объекты; 
    \item .NET Core и .NET Standard -- это современные, развивающиеся и востребованные кроссплатформенные технологии с открытым исходным кодом;
    \item Личные предпочтения автора.
\end{itemize}
Для поэтапного создания программы, были сформулированы и решены следующие задачи:
\begin{itemize}
    \item Разработать библиотеку классов для объектов языка элементарной алгебры: символы алфавита, термы, формулы;
    \item Реализовать ввод логических формул;
    \item Разработать библиотеку классов для рациональных чисел и многочленов от одной вещественной переменной с рациональными коэффициентами;
    \item Реализовать алгоритм Тарского для базового случая;
    \item Реализовать алгоритм элиминации кванторов(ссылка на утверждение о сведении задачи к алгоритму тарского) для простых формул языка элементарной алгебры;
    \item Обеспечить минимальное покрытие юнит-тестами;
    \item Собрать все библиотеки и модули в единый программный комплекс.
\end{itemize}
Написание программы осуществлялось в среде разработки Microsoft Visual Studio 2019, доступной для использования в некоммерческих целях и скачивания на официальном сайте: https://visualstudio.microsoft.com/ru/.

Также была рассмотрена перспектива реализации алгоритма Тарского для общего случая. 

\subsection{Представление формул. Библиотека LogicLanguageLib}
форма представления символов и формул языка

\subsection{Система ввода. Класс Parser}
Ввод формул

\subsection{Реализация базового случая. Библиотека SimpleTarskiAlgorithmLib}
Часть математики, часть алгоритм. Runner. Покрытие тестами. Консольное приложение

\subsection{Библиотека SimpleTarskiAlgorithmRunner}
SimpleTarskiAlgorithmRunner

\subsection{Юнит-тестирование}
тесты

\subsection{Результаты работы}
Демонстрация результатов

\subsection{Общий случай}
Часть математики. Описание нерешённых задач.




    \input{заключение.tex}

    \input{список литературы.tex}
    
    \section*{Приложение А}
	\addcontentsline{toc}{section}{Приложение А}

Приложение	

\end{document}