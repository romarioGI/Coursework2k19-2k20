\section{Программная реализация}

Для реализации программы, которая по простой формуле (определение \ref{simpleFormula}) языка элементарной алгебры, вводимой пользователем с клавиатуры, строит эквивалентную бескванторную формулу, были выбраны язык C\#, платформа .NET Core 3.1 и спецификация .NET Standard 2.1 \cite{TroelsonNet}. Такой выбор обусловлен рядом причин:
\begin{itemize}
    \item Язык C\# -- это объектно-ориентированный язык программирования, а данная парадигма программирования позволяет абстрактно описывать объекты, в том числе и математические объекты; 
    \item .NET Core и .NET Standard -- это современные, развивающиеся и востребованные кроссплатформенные технологии с открытым исходным кодом;
    \item Личные предпочтения автора.
\end{itemize}
Для поэтапного создания программы, были сформулированы и решены следующие задачи:
\begin{itemize}
    \item Разработать библиотеку классов для объектов языка элементарной алгебры: символы алфавита, термы, формулы;
    \item Реализовать ввод логических формул;
    \item Разработать библиотеку классов для рациональных чисел и многочленов от одной вещественной переменной с рациональными коэффициентами;
    \item Реализовать алгоритм Тарского для базового случая;
    \item Реализовать алгоритм элиминации кванторов(ссылка на утверждение о сведении задачи к алгоритму тарского) для простых формул языка элементарной алгебры;
    \item Обеспечить минимальное покрытие юнит-тестами;
    \item Собрать все библиотеки и модули в единый программный комплекс.
\end{itemize}
Написание программы осуществлялось в среде разработки Microsoft Visual Studio 2019, доступной для использования в некоммерческих целях и скачивания на официальном сайте: https://visualstudio.microsoft.com/ru/.

Также была рассмотрена перспектива реализации алгоритма Тарского для общего случая. 

\subsection{Представление формул. Библиотека LogicLanguageLib}
форма представления символов и формул языка

\subsection{Система ввода. Класс Parser}
Ввод формул

\subsection{Реализация базового случая. Библиотека SimpleTarskiAlgorithmLib}
Часть математики, часть алгоритм. Runner. Покрытие тестами. Консольное приложение

\subsection{Библиотека SimpleTarskiAlgorithmRunner}
SimpleTarskiAlgorithmRunner

\subsection{Юнит-тестирование}
тесты

\subsection{Результаты работы}
Демонстрация результатов

\subsection{Общий случай}
Часть математики. Описание нерешённых задач.


