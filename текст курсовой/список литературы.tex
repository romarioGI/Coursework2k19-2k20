\begin{thebibliography}{3}
    \bibitem{Gibadulin1}
    Гибадулин Р. А. Алгоритм поиска вывода в Исчислении Высказываний и его программная реализация // Современные проблемы математики и информатики : сборник научных трудов молодых ученых, аспирантов и студентов. / Яросл. гос. ун-т им. П. Г. Демидова. -- Ярославль : ЯрГУ, 2019. -- Вып. 19. -- С. 28 -- 37.

    \bibitem{DurnevML}
    Дурнев, В. Г. Элементы теории множеств и математической логики: учеб. пособие / Яросл. гос. ун-т. им. П. Г. Демидова, Ярославль, 2009 --- 412 с.

    \bibitem{Matiyasevich}
    Матиясевич, Ю. В. Алгоритм Тарского // Компьютерные инструменты в образовании. -- 2008. -- № 6. -- С. 14.

    \bibitem{TroelsonNet}
    Троелсен, Э. Язык программирования С\# 7 и платформы .NET и .NET Core / Э. Троелсен, Ф. Джепикс; пер. с англ. и ред. Ю.Н. Артеменко. -- 8-е изд. -- М.; СПб.: Диалектика, 2020. -- 1328 с.

    \bibitem{YakimovaYaP2}
    Якимова, О. П. Языки программирования. Ч.2: лабораторный практикум / О. П. Якимова, И. М. Якимов, В. Л. Дольников; Яросл. гос. ун-т им. П. Г. Демидова. -- Ярославль : ЯрГУ, 2012. -- 56 с.  

    \bibitem{Tarski}
    Tarski, A. A Decision Method for Elementary Algebra and Geometry: Prepared for Publication with the Assistance of J.C.C. McKinsey, Santa Monica, Calif.: RAND Corporation, R-109, 1951. 

\end{thebibliography}

Добавить учебник по матану и алгебре

\addcontentsline{toc}{section}{\refname}