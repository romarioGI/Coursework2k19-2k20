\section{Алгоритм Тарского}

Прежде всего определим область математики, истинность утверждений которой должен проверять алгоритм. Затем формально опишем язык, на котором записываются эти утверждения. И, наконец, опишем алгоритм, который по формуле описанного языка строит эквивалентную бескванторную формулу.

\subsection{Элементарная алгебра}

Под элементарной алгеброй понимается та часть общей теории действительных чисел, в которой используются переменные, представляющие собой действительные числа, и константы для всех рациональных чисел, определены арифметические операции, такие как <<сложение>> и <<умножение>>, и отношения сравнения действительных чисел~--- <<меньше>>, <<больше>> и <<равно>>. То есть рассматриваются системы алгебраических уравнений и неравенств.

Заметим, что используя декартову систему координат, некоторые задачи геометрии можно сформулировать как задачи элементарной алгебры. Например, теорема о пересечении высот треугольника, которая утверждает, что три высоты невырожденного треугольника пересекаются в одной точке, равносильна утверждению: для любых трех точек $A(x_1, y_1)$, $B(x_2, y_2)$ и $C(x_3, y_3)$, не лежащих на одной прямой, существует точка $D(x_4, y_4)$ такая, что $\overrightarrow{AD} \perp \overrightarrow{BC}$, $\overrightarrow{BD} \perp \overrightarrow{AC}$ и $\overrightarrow{CD} \perp \overrightarrow{AB}$. Иначе говоря, если $\overrightarrow{AB} \land \overrightarrow{AC} \neq 0$, то система
\begin{equation*}
    \begin{cases}
        \left(\overrightarrow{AD}, \overrightarrow{BC}\right) = 0 \\
        \left(\overrightarrow{BD}, \overrightarrow{AC}\right) = 0 \\
        \left(\overrightarrow{CD}, \overrightarrow{AB}\right) = 0
    \end{cases}
\end{equation*}
имеет решение относительно переменных $x_4$, $y_4$, где $* \land *$~--- псевдоскалярное произведение векторов, $(*, *)$~--- скалярное произведение векторов.

\subsection{Язык элементарной алгебры}

Язык элементарной алгебры~--- это язык логики первого порядка с сигнатурой
\begin{equation*}
    \tau = \langle \, \mathbb{Q},\, F,\, P,\, \theta,\, \phi \, \rangle,
\end{equation*}
где $\mathbb{Q}$~--- множество рациональных чисел, которое является множеством индивидных констант, $F =  \left\{+,\, \cdot\right\}$~--- множество функциональных символов, $P = \left\{ <,\, >,\, = \right\}$~--- множество предикатных символов, $\theta : F \rightarrow \mathbb{N}$ такое, что $\theta(+) = 2$ и $\theta(\cdot) = 2$, и $\phi : P \rightarrow \mathbb{N}$ такое, что $\phi(<) = 2$, $\phi(>) = 2$ и $\phi(=) = 2$. Из определения отображений $\theta$ и $\phi$ следует, что все $f \in F$ являются двухместными функциональными символами, а все $p \in P$ являются двухместными предикатными символами. Основное множество интерпретации языка $L_\tau $ совпадает с множеством действительных чисел $\mathbb{R}$, отображение множества индивидных констант в основное множество определяется естественным образом, так как $\mathbb{Q} \subset \mathbb{R}$, функциональные символы $+$ и $\cdot$ отображаются в сложение и умножение в поле $\mathbb{R}$ соответственно, и предикатные символы $<$, $>$ и $=$ отображаются естественным образом в операции сравнения в $\mathbb{R}$.

\begin{remark}
    Множество констант ограничено рациональными числами лишь потому, что компьютер может быстро работать с ними без потери точности, что нельзя сказать про действительные числа.
\end{remark}

Например, теорема о пересечении высот на языке элементарной алгебры записывается так:
\begin{gather*}
    (\forall x_1)(\forall y_1)(\forall x_2)(\forall y_2)(\forall x_3)(\forall y_3) \\
    (
        (
            \lnot
            (
                (x_2 - x_1)\cdot(y_3 - y_1) - (y_2 - y_1)\cdot(x_3 - x_1) = 0
            )
        )
        \to \\
        (\exists x_4)(\exists y_4)
        (
           (
                (x_4 - x_3)\cdot(x_2 - x_1) + (y_4 - y_3)\cdot(y_2 - y_1) = 0
            ) \& \\
            (
                (x_4 - x_2)\cdot(x_1 - x_3) + (y_4 - y_2)\cdot(y_1 - y_3) = 0
            ) \& \\
            (
                (x_4 - x_1)\cdot(x_3 - x_2) + (y_4 - y_1)\cdot(y_3 - y_2) = 0
            )
        )
    ).
\end{gather*}

\begin{remark}
    Нет необходимости формально вводить такие операции как вычитание, деление и возведение в степень по следующим соображениям:
    \begin{equation*}
        a-b=a+(-1)\cdot b; \quad \frac{a}{b} > 0 \Leftrightarrow (a > 0 \, \& \, b > 0) \lor (a < 0 \, \& \, b < 0); \quad x^2 = x \cdot x.
    \end{equation*}
\end{remark}

\subsection{Элиминация кванторов}

\begin{definition}
    \textbf{Элиминация кванторов}~--- это процесс, порождающий по заданной логической формуле, другую, эквивалентную ей бескванторную формулу, то есть свободную от вхождений кванторов.
\end{definition}

Пусть алгоритм $A$ такой, что $A\left(\left(Qx\right)\mathcal{A}\right) = \mathcal{B}$, где $\mathcal{A}$ и $\mathcal{B}$~--- бескванторные формулы языка элементарной алгебры, и формулы $(Qx)\mathcal{A}$ и $\mathcal{B}$ эквивалентны, а $Q$~--- квантор. Тогда верно следующее утверждение:
\begin{proposal}\label{algB}
    Если алгоритм $A$ существует, то существует алгоритм $B$ такой, что для любой формулы $\mathcal{A}$ языка элементарной алгебры $B\left(\mathcal{A}\right)$~--- бескванторная формула, эквивалентная $\mathcal{A}$.
\end{proposal}   
\begin{proof}
    Определим алгоритм $B$ следующим образом:
    \begin{itemize}
        \item Если $\mathcal{A}$~--- бескванторная формула, то $B\left(\mathcal{A}\right) = \mathcal{A}$;
        \item Если $\mathcal{A} = \left(Qx\right)\mathcal{B}$, то $B\left(\mathcal{A}\right) = A\left(\left(Qx\right)B\left(\mathcal{B}\right)\right)$. Формула $B\left(\mathcal{B}\right)$~--- бескванторная по построению $B$, следовательно запись $A\left(\left(Qx\right)B\left(\mathcal{B}\right)\right)$ корректна, при этом $B\left(\mathcal{B}\right)$ эквивалентна $\mathcal{B}$, следовательно, $\left(Qx\right)\mathcal{B}$ эквивалентна $\left(Qx\right)B\left(\mathcal{B}\right)$, а значит $\mathcal{A}$ эквивалентна $B\left(\mathcal{A}\right)$. Также заметим, что длина формулы $\mathcal{B}$ строго меньше длины формулы $\mathcal{A}$;
        \item Если $\mathcal{A}$ не удовлетворяет предыдущим условиям, то
        \begin{itemize}
            \item либо $\mathcal{A} = \lnot \mathcal{B}$, тогда $B\left(\mathcal{A}\right) = \lnot B\left(\mathcal{B}\right)$,
            \item либо $\mathcal{A} = \mathcal{B} * \mathcal{C}$, тогда $B\left(\mathcal{A}\right) = B\left(\mathcal{B}\right) * B\left(\mathcal{C}\right)$, где $* \in \left\{\lor, \&, \to\right\}$.
        \end{itemize}
        При этом длины формул $\mathcal{B}$ и $\mathcal{C}$ меньше длины формулы $\mathcal{A}$.
    \end{itemize}
    Алгоритм $B$ определен рекурсивно, при этом на каждом этапе на вход $B$ подаётся формула меньшей длины, следовательно, алгоритм $B$ является конечным, и на каждом шаге выход алгоритма~--- бескванторная эквивалентная формула.
\end{proof} 
\begin{remark}
    Данное утверждение верно и для других языков логики предикатов.
\end{remark}

Таким образом, для элиминации кванторов произвольной формулы достаточно построить алгоритм $A$ и применить описанный в утверждении \ref{algB} алгоритм $B$.

\subsection{Алгоритм Тарского}

Термы в языке элементарной алгебры~--- это многочлены с рациональными коэффициентами от действительных переменных Тогда очевидно, что выражения
\begin{equation*}
    f < g, \quad f = g, \quad f > g
\end{equation*}
равносильны выражениям
\begin{equation*}
    f - g < 0, \quad f - g = 0, \quad f - g > 0 
\end{equation*}
соответственно, где $f$ и $g$~--- термы. Поэтому, не нарушая общности рассуждений, можно считать, что все атомарные формулы имеют вид:
\begin{equation*}
    f < 0, \quad f = 0, \quad f > 0.
\end{equation*}
Поэтому нас будет интересовать только знак многочлена.

\begin{remark}
    На данном этапе можно считать, что все рассматриваемые многочлены ненулевые, так как знак нулевого многочлена в любой точке определяется тривиальным образом.
\end{remark}

Рассмотрим формулу $\mathcal{A} = (Qx)(f(x) \, \rho \, 0)$, где $Q$~--- квантор, $f(x)$~--- многочлен от одной переменной, $\rho$~--- предикат. Известно, что многочлены от одной переменной сохраняют свой знак, то есть на интервалах между корнями значение предиката неравенства или равенства с нулем постоянно. Следовательно, чтобы уметь определять знак значения многочлена в произвольной точке, достаточно знать значения многочлена лишь в \textbf{конечном} наборе точек~--- во всех корнях, в каких-то точках между любой парой соседних корней, а также в точках, одна из которых заведомо правее, а другая~--- левее всех корней. Тогда формуле $\mathcal{A}$ эквивалентна следующая бескванторная формула:
\begin{equation*}
    \mathcal{B} = 
    \begin{cases}
        \bigvee\limits_{x_0 \in X} (f(x_0) \, \rho \, 0), &\text{если $Q = \exists$} \\
        \bigwith\limits_{x_0 \in X} (f(x_0) \, \rho \, 0), &\text{если $Q = \forall$} \\
    \end{cases}
\end{equation*}
где $X$~--- конечное множество этих точек.

Рассмотрим формулу $\mathcal{A} = (Qx)(\Phi(x))$, где $\Phi(x)$~--- бескванторная формула, которая может содержать вхождения лишь переменной $x$. Аналогично предыдущему случаю, пусть множество $X$ состоит из корней многочленов, входящих в формулу $\Phi(x)$, и точек, выбранных между парами соседних корней, а также пусть в это множество входит точка, которая правее всех корней, и точка, которая левее всех корней. Тогда формула
\begin{equation*}
    \mathcal{B} = 
    \begin{cases}
        \bigvee\limits_{x_0 \in X} \Phi(x_0), &\text{если $Q = \exists$} \\
        \bigwith\limits_{x_0 \in X} \Phi(x_0), &\text{если $Q = \forall$} \\
    \end{cases}
\end{equation*}
свободна от вхождений кванторов и эквивалента формуле $\mathcal{A}$.

По теореме Ролля о нуле производной, для любой пары корней $x_1, x_2$ между ними существует такая точка $\xi$, что производная многочлена в точке $\xi$ обращается в ноль. Поэтому в качестве точек между корнями можно использовать корни производной этого многочлена.

Аналогично, множество корней многочлена
\begin{equation*}
    \prod\limits_{i = 1}^n f_i(x)
\end{equation*}
совпадает с объединением множеств корней многочленов $f_1(x), ... , f_n(x)$, тогда в качестве точек между корнями можно рассматривать корни многочлена
\begin{equation*}
    f_0(x) = \left(\prod\limits_{i = 1}^n f_i(x)\right)^\prime. 
\end{equation*}

\subsubsection{Таблица Тарского}

Упорядочим выбранные точки $X = \left\{x_1, ... , x_s\right\}$ по возрастанию и запишем значения многочленов в этих точках в таблицу:
\begin{center}
    \begin{tabular}{ |c|c|c|c|c|c| } 
    \hline
                 & $x_1$ & $...$ & $x_j$ & $...$ & $x_s$ \\ 
    \hline
        $f_1$ & $f_1(x_1)$ & $...$ & $f_1(x_j)$ & $...$ & $f_1(x_s)$\\ 
    \hline
        $\vdots$ & $\vdots$ & $\ddots$ & $\vdots$ & $\ddots$ & $\vdots$ \\
    \hline
        $f_i$ & $f_i(x_1)$ & $...$ & $f_i(x_j)$ & $...$ & $f_i(x_s)$\\ 
    \hline
        $\vdots$ & $\vdots$ & $\ddots$ & $\vdots$ & $\ddots$ & $\vdots$ \\
    \hline
        $f_n$ & $f_n(x_1)$ & $...$ & $f_n(x_j)$ & $...$ & $f_n(x_s)$\\    
    \hline
    \end{tabular}
\end{center}
Так как нас интересуют только знаки многочленов в этих точках, то достаточно записывать лишь символ знака значения:
\begin{center}
    \begin{tabular}{ |c|c|c|c|c|c| } 
    \hline
                 & $x_1$ & $...$ & $x_j$ & $...$ & $x_s$ \\ 
    \hline
        $f_1$ & $\varepsilon_{11}$ & $...$ & $\varepsilon_{1j}$ & $...$ & $\varepsilon_{1s}$\\ 
    \hline
        $\vdots$ & $\vdots$ & $\ddots$ & $\vdots$ & $\ddots$ & $\vdots$ \\
    \hline
        $f_i$ & $\varepsilon_{i1}$ & $...$ & $\varepsilon_{ij}$ & $...$ & $\varepsilon_{is}$\\ 
    \hline
        $\vdots$ & $\vdots$ & $\ddots$ & $\vdots$ & $\ddots$ & $\vdots$ \\
    \hline
        $f_n$ & $\varepsilon_{n1}$ & $...$ & $\varepsilon_{nj}$ & $...$ & $\varepsilon_{ns}$\\    
    \hline
    \end{tabular}
\end{center}
где 
\begin{equation*}
    \varepsilon_{ij} = 
    \begin{cases}
        +, &\text{если $f_i(x_j) > 0$} \\
        0, &\text{если $f_i(x_j) = 0$} \\
        -, &\text{если $f_i(x_j) < 0$}
    \end{cases}.
\end{equation*}
Таблицы такого вида будем называть \textbf{таблицами Тарского}. 

\begin{proposal}
    Знаки $+$ и $-$ не могут стоять в двух соседних по горизонтали клетках таблицы Тарского.
\end{proposal}
\begin{proof}
    Следует из теоремы Коши о нулях непрерывной функции, непрерывности многочленов как функций и упорядоченности точек $x_1, ..., x_s$.
\end{proof}

Имея таблицу Тарского, нетрудно вычислить истинностное значение формулы $\Phi(x_i)$, так как для определения истинностного значения атомарной формулы достаточно посмотреть на соответствующую клетку таблицы. При этом уже нет необходимости знать точки $x_1, ..., x_n$. А зная истинностное значение формулы, можно построить эквивалентную бескванторную:
\begin{itemize}
    \item если формула истинна~--- то можно использовать любую тождественно истинную формулу, например, $0=0$;
    \item если формула ложна~--- то можно использовать любую тождественно ложную формулу, например $0=1$.
\end{itemize}

До сих пор не обсуждалось, как искать корни многочленов. Оказывается, таблицу Тарского можно построить не находя ни одного корня, если рассмотреть системы многочленов особого вида.

\subsubsection{Насыщенная система}

\begin{definition}\cite{lect1}
    Система функций называется \textbf{полунасыщенной}, если вместе с каждой функцией, отличной от постоянной функции, она содержит и ее производную.
\end{definition}

\begin{proposal}\cite{lect1}
    Каждую конечную систему многочленов можно расширить до конечной полунасыщенной системы.
\end{proposal}

\begin{proposal}\label{two zero}\cite{lect1}
    Если многочлен отличен от тождественного нуля, то в строке таблицы Тарского, соответствующей этому многочлену, в соседних по горизонтали клетках не могут стоять два символа 0.
\end{proposal}

\begin{definition}\cite{lect1}
    Полунасыщенная система многочленов $p_1(x), ... , p_n(x)$ называется \textbf{насыщенной}, если вместе с каждыми двумя многочленами $p_i(x)$ и $p_j(x)$ такими, что $0 < deg(p_j(x)) \leq deg(p_i(x))$, она содержит и остаток $r(x)$ от деления $p_i(x)$ на $p_j(x)$. 
\end{definition}

\begin{proposal}\cite{lect1}
    Каждую конечную систему многочленов можно расширить до конечной насыщенной системы.
\end{proposal}

\begin{proposal}\label{subsystem}\cite{lect1}
    Если $p_1(x), ... , p_{n-1}(x), p_n(x)$~--- насыщенная система многочленов, и 
    \begin{equation*}
        deg(p_1(x)) \leq ... \leq deg(p_{n-1}(x)) \leq deg(p_n(x)),
    \end{equation*}
    то система $p_1(x), ... , p_{n-1}(x)$ также является насыщенной.
\end{proposal}

\begin{proposal}\label{min deg}
    Если $p_1(x), ... , p_n(x)$~--- насыщенная система многочленов, и 
    \begin{equation*}
        deg(p_1(x)) \leq deg(p_i(x)),\,\text{где $i = 2, 3, ... , n$},
    \end{equation*}
    то $deg(p_1(x)) < 1$.
\end{proposal}
\begin{proof}
    Если предположить противное, то с одной стороны, система должна содержать многочлен $p_1^\prime(x)$, степень которого меньше $deg(p_1(x))$, а с другой стороны, степени всех многочленов должны быть не меньше $deg(p_1(x))$, противоречие.
\end{proof}

\subsubsection{Метод построения таблицы Тарского}

Пусть $p_1(x), ... , p_{n-1}(x), p_n(x)$~--- насыщенная система многочленов, и многочлены упорядочены в ней по не убыванию степени.

Рассмотрим подсистему из одного элемента $p_1(x)$. Согласно утверждению \ref{subsystem}, система $p_1(x)$ является насыщенной, тогда многочлен $p_1(x)$ представляет собой константу, так как его степень меньше единицы, согласно утверждению \ref{min deg}. В таком случае знак многочлена в любой точке совпадает со знаком этой константы. Таблица Тарского для одного многочлена имеет вид:
\begin{center}
    \begin{tabular}{ |c|c|c| } 
    \hline
        & $-\infty$ & $+\infty$ \\ 
    \hline
        $p_1$ & $\varepsilon$ & $\varepsilon$\\ 
    \hline
    \end{tabular}
\end{center}
Символами $-\infty$ и $+\infty$ обозначены точки, которые заведомо расположены левее и правее всех корней соответственно. Выбирать конкретные значения для этих точек не нужно, потому что в точке $+\infty$ знак многочлена совпадает со знаком старшего коэффициента, а в точке $-\infty$ знак зависит от четности степени многочлена:
\begin{itemize}
    \item если четная, то совпадает со знаком старшего коэффициента;
    \item иначе равен знаку противоположному к знаку старшего коэффициента.
\end{itemize}
В таблице всего два столбца, поэтому верно утверждение: для каждого столбца $j$, за исключением самого правого и самого левого, в этой таблице существует ненулевой многочлен $p_i(x)$ такой, что $\varepsilon_{i, j} = 0$.

Индуктивное предположение: пусть для насыщенной системы $p_1(x), ... , p_{k-1}(x)$ уже построена таблица Тарского:
\begin{center}
    \begin{tabular}{ |c|c|c|c|c|c| } 
    \hline
                 & $-\infty$ & $...$ &  & $...$ & $+\infty$ \\ 
    \hline
        $p_1$ & $\varepsilon_{1, 1}$ & $...$ & $\varepsilon_{1, j}$ & $...$ & $\varepsilon_{1, s}$\\ 
    \hline
        $\vdots$ & $\vdots$ & $\ddots$ & $\vdots$ & $\ddots$ & $\vdots$ \\
    \hline
        $p_{k-1}$ & $\varepsilon_{k-1, 1}$ & $...$ & $\varepsilon_{k-1, j}$ & $...$ & $\varepsilon_{k-1, s}$\\    
    \hline
    \end{tabular}
\end{center}
И для каждого столбца $j$, за исключением самого правого и самого левого, в этой таблице существует ненулевой многочлен $p_i(x)$ такой, что $\varepsilon_{i, j} = 0$.

К этой таблице добавим строку для многочлена $p_{k}(x)$, записав знаки для крайних столбцов.
\begin{center}
    \begin{tabular}{ |c|c|c|c|c|c| } 
    \hline
                 & $-\infty$ & $...$ &  & $...$ & $+\infty$ \\ 
    \hline
        $p_1$ & $\varepsilon_{1, 1}$ & $...$ & $\varepsilon_{1, j}$ & $...$ & $\varepsilon_{1, s}$\\ 
    \hline
        $\vdots$ & $\vdots$ & $\ddots$ & $\vdots$ & $\ddots$ & $\vdots$ \\
    \hline
        $p_{k-1}$ & $\varepsilon_{k-1, 1}$ & $...$ & $\varepsilon_{k-1, j}$ & $...$ & $\varepsilon_{k-1, s}$\\    
    \hline
        $p_{k}$ & $\varepsilon_{k, 1}$ & $...$ & $ $?$ $ & $...$ & $\varepsilon_{k, s}$\\    
    \hline
    \end{tabular}
\end{center}
Для каждого столбца $j$ рассмотрим многочлен $p_i(x)$ такой, что $\varepsilon_{i, j} = 0$. Этот многочлен существует и отличен от тождественного нуля в силу индуктивного предположения.
\begin{proposal}\label{остаток}
    Пусть $f(x)$ и $g(x)$~--- ненулевые многочлены. Если $g(a) = 0$, то f(a) = r(a), где r(x)~--- остаток от деления многочлена f(x) на g(x).
\end{proposal}
\begin{proof}
    Многочлен $r(x)$~--- остаток от деления, тогда $f(x) = q(x)g(x) + r(x)$, подставив $a$ получим $f(a) = q(a)g(a) + r(a) = q(a)\cdot 0 + r(a) = r(a)$.
\end{proof}

Найдём $p_t(x)$~--- остаток от деления $p_k(x)$ на $p_i(x)$. Система многочленов насыщена, поэтому многочлен $p_t(x)$ уже добавлен в таблицу, следовательно, по утверждению \ref{остаток}, $\varepsilon_{k, j} = \varepsilon_{t, j}$. Таким образом, заполняется вся нижняя строка.

Просмотрим значения в нижней строке. Может случиться так, что в соседних клетках стоят знаки $+$ и $-$. В таком случае необходимо добавить новый столбец между теми столбцами, в которых находятся эти клетки. Понятно, что в новом столбце нижняя клетка заполняется нулем, что следует из теоремы Коши о нулях непрерывной функции. Для оставшихся клеток рассмотрим случаи.
\begin{center}
    \begin{tabular}{|c|c|c|}
        \hline
        $+$ & ? & $+$\\
        \hline
    \end{tabular}
        \quad
    \begin{tabular}{|c|c|c|}
        \hline
        $+$ & ? & $0$\\
        \hline
    \end{tabular}           
\end{center}
Тогда вместо символа $?$ ставится знак $+$.
\begin{center}
    \begin{tabular}{|c|c|c|}
        \hline
        $0$ & ? & $+$\\
        \hline
    \end{tabular}
        \quad
    \begin{tabular}{|c|c|c|}
        \hline
        $0$ & ? & $-$\\
        \hline
    \end{tabular}           
\end{center}
В этих случаях ставится знак $+$ или $-$ соответственно.
\begin{center}
    \begin{tabular}{|c|c|c|}
        \hline
        $-$ & ? & $0$\\
        \hline
    \end{tabular}
        \quad
    \begin{tabular}{|c|c|c|}
        \hline
        $-$ & ? & $-$\\
        \hline
    \end{tabular}           
\end{center}
В этих~--- знак $-$.
И наконец, в случае 
\begin{center}
    \begin{tabular}{|c|c|c|}
        \hline
        $0$ & ? & $0$\\
        \hline
    \end{tabular} 
\end{center}
ставится $0$, так как эта строка точно соответствует нулевому многочлену.

А случаи
\begin{center}
    \begin{tabular}{|c|c|c|}
        \hline
        $+$ & ? & $-$\\
        \hline
    \end{tabular}
        \quad
    \begin{tabular}{|c|c|c|}
        \hline
        $-$ & ? & $+$\\
        \hline
    \end{tabular}      
\end{center}
невозможны по построению таблицы Тарского.

Таким образом, удалось построить таблицу Тарского для насыщенной системы многочленов $p_1(x), ... , p_{k-1}(x), p_k(x)$, при этом для каждого столбца найдется многочлен, на пересечении строки которого с выбранным столбцом в клетке записан символ $0$.

\subsubsection{Алгоритм для формулы вида $(Qx)\Phi(x)$}

Все готово, чтобы описать алгоритм Тарского для формулы $\mathcal{A} = (Qx)\Phi(x)$:
\begin{enumerate}
    \item Составить список всех многочленов $f_1(x), ... , f_n(x)$, входящих в $\Phi(x)$ и отличных от тождественного нуля;
    \item Добавить к этому списку многочлен 
    \begin{equation*}
        f_0(x) = \left( \prod\limits_{i = 1}^n f_i(x) \right)^\prime;
    \end{equation*}
    \item Расширить этот список до насыщенной системы $p_1(x), ... , p_m(x)$, упорядоченной по не убыванию степени;
    \item Построить таблицу Тарского, по очереди добавляя многочлены; 
    \item Вычислить истинностное значение $\Phi(x)$ для каждого из столбцов таблицы;
    \item Если $Q = \exists$, то формула $\mathcal{A}$ истинна тогда и только тогда, когда хотя бы одно из вычисленных значений истинно. Если $Q = \forall$, то формула $\mathcal{A}$ истинна тогда и только тогда, когда все вычисленные значения истинны.
\end{enumerate}

\subsubsection{Алгоритм для формулы вида $(Qx)\Phi(x, a_1, ... , a_l)$}

Оказывается, в случае, когда формула имеет вид $(Qx)\Phi(x, a_1, ... , a_l)$, нужно лишь немного модифицировать алгоритм. Во-первых, коэффициенты многочленов теперь не из $\mathbb{Q}$, а из поля частных целостного кольца $\mathbb{Q}\left[a_1, ... , a_l\right]$. Во-вторых, нельзя говорить о знаках таких коэффициентов, поэтому каждый раз, когда необходимо определить знак коэффициента, придется разбирать три случая: коэффициент меньше нуля, больше нуля или равен нулю. Поэтому будет построено дерево разбора случаев. В листьях этого дерева все знаки определены и можно построить таблицу Тарского. Если по таблице получается, что формула истинна, тогда все предположения, сделанные в ходе разбора случаев, выписываются в виде конъюнкции. Результатом же работы алгоритма будет дизъюнкция всех таких конъюнкций.

\subsubsection{Пример работы алгоритма}
Рассмотрим формулу $(\forall x)(y < 0 \, \to \, x^2 > y)$ и построим эквивалентную ей бескванторную формулу с помощью алгоритма Тарского. Многочлены $y$ и $x^2 - y$ входят в данную формулу. Далее нужно выяснить все ли многочлены отличны от тождественного нуля, поэтому рассмотрим два случая:
\begin{enumerate}
    \item $y = 0$, тогда из списка исключается многочлен $y \equiv 0$ и добавляется многочлен $2x$, при этом $x^2 - y \equiv x^2$;
    \item $y < 0$ или $y > 0$, тогда в систему добавляется многочлен $2yx$.
\end{enumerate}
Переходим к насыщению системы:
\begin{enumerate}
    \item система $2x, x^2$ дополняется до $0, 2, 2x, x^2$;
    \item система $y, 2yx, x^2 - y$ дополняется до $0, y, -y, 2y, 2, 2yx, 2x, x^2 - y$.
\end{enumerate}
Построим таблицы Тарского:
\begin{enumerate}
    \item $y = 0$, система $0, 2, 2x, x^2$:
    \begin{center}
        \begin{tabular}{|c|c|c|}
            \hline
             & $-\infty$ & $+\infty$\\
            \hline
            $0$ & $0$ & $0$\\
            \hline
            $2$ & $+$ & $+$\\
            \hline
        \end{tabular} 
            \quad
        \begin{tabular}{|c|c|c|c|}
            \hline
             & $-\infty$ & & $+\infty$\\
            \hline
            $0$ & $0$ & $0$ & $0$\\
            \hline
            $2$ & $+$ & $+$ & $+$\\
            \hline
            $2x$ & $-$ & $0$ & $+$\\
            \hline
        \end{tabular}
            \quad
        \begin{tabular}{|c|c|c|c|}
            \hline
             & $-\infty$ & & $+\infty$\\
            \hline
            $0$ & $0$ & $0$ & $0$\\
            \hline
            $2$ & $+$ & $+$ & $+$\\
            \hline
            $2x$ & $-$ & $0$ & $+$\\
            \hline
            $x^2$ & $+$ & $0$ & $+$\\
            \hline
        \end{tabular}         
    \end{center}
    \item $y < 0$, система $0, y, -y, 2y, 2, 2yx, 2x, x^2 - y$:
    \begin{center}
        \begin{tabular}{|c|c|c|}
            \hline
             & $-\infty$ & $+\infty$\\
            \hline
            $0$ & $0$ & $0$\\
            \hline
            $y$ & $-$ & $-$\\
            \hline
            $-y$ & $+$ & $+$\\
            \hline
            $2y$ & $-$ & $-$\\
            \hline
            $2$ & $+$ & $+$\\
            \hline
        \end{tabular} 
            \quad
        \begin{tabular}{|c|c|c|c|}
            \hline
             & $-\infty$ & & $+\infty$\\
            \hline
            $0$ & $0$ & $0$ & $0$\\
            \hline
            $y$ & $-$ & $-$ & $-$\\
            \hline
            $-y$ & $+$ & $+$ & $+$\\
            \hline
            $2y$ & $-$ & $-$ & $-$\\
            \hline
            $2$ & $+$ & $+$ & $+$\\
            \hline
            $2yx$ & $+$ & $0$ & $-$\\
            \hline
            $2x$ & $-$ & $0$ & $+$\\
            \hline
            $x^2 - y$ & $+$ & $+$ & $+$\\
            \hline
        \end{tabular} 
            \quad    
    \end{center}
    \item $y > 0$, система $0, y, -y, 2y, 2, 2yx, 2x, x^2 - y$:
    \begin{center}
        \begin{tabular}{|c|c|c|}
            \hline
             & $-\infty$ & $+\infty$\\
            \hline
            $0$ & $0$ & $0$\\
            \hline
            $y$ & $+$ & $+$\\
            \hline
            $-y$ & $-$ & $-$\\
            \hline
            $2y$ & $+$ & $+$\\
            \hline
            $2$ & $+$ & $+$\\
            \hline
        \end{tabular} 
            \quad
        \begin{tabular}{|c|c|c|c|}
            \hline
             & $-\infty$ & & $+\infty$\\
            \hline
            $0$ & $0$ & $0$ & $0$\\
            \hline
            $y$ & $+$ & $+$ & $+$\\
            \hline
            $-y$ & $-$ & $-$ & $-$\\
            \hline
            $2y$ & $+$ & $+$ & $+$\\
            \hline
            $2$ & $+$ & $+$ & $+$\\
            \hline
            $2yx$ & $-$ & $0$ & $+$\\
            \hline
            $2x$ & $-$ & $0$ & $+$\\
            \hline
        \end{tabular} 
            \quad 
        \begin{tabular}{|c|c|c|c|c|c|}
            \hline
             & $-\infty$ & & & & $+\infty$\\
            \hline
            $0$ & $0$ & $0$ & $0$ & $0$ & $0$\\
            \hline
            $y$ & $+$ & $+$ & $+$ & $+$ & $+$\\
            \hline
            $-y$ & $-$ & $-$ & $-$ & $-$ & $-$\\
            \hline
            $2y$ & $+$ & $+$ & $+$ & $+$ & $+$\\
            \hline
            $2$ & $+$ & $+$ & $+$ & $+$ & $+$\\
            \hline
            $2yx$ & $-$ & $-$ & $0$ & $+$ & $+$\\
            \hline
            $2x$ & $-$ & $-$ & $0$ & $+$ & $+$\\
            \hline
            $x^2 - y$ & $+$ & $0$ & $-$ & $0$ & $+$\\
            \hline
        \end{tabular} 
            \quad       
    \end{center}
\end{enumerate}
Нетрудно убедиться в том, что для каждого случая, для каждого столбца формула $(y < 0 \, \to \, x^2 > y)$ истинна. В результате, на выходе алгоритма получим формулу
\begin{equation*}
    (y = 0 \lor y < 0 \lor y > 0).
\end{equation*}

\subsection{Теорема Тарского}

\begin{theorem}[Альфред Тарский]
    Для любой формулы $\mathcal{A}$ языка элементарной алгебры существует эквивалентная ей бескванторная формула этого же языка.
\end{theorem}


Алгоритм, предложенный А. Тарским в его работе \cite{Tarski}, записывался иначе~--- предельно формально. Но и цель была не предложить <<хороший>> алгоритм, а доказать, что элементарная алгебра допускает элиминацию кванторов. В последующие годы велась работа по упрощению и усовершенствованию алгоритма, особенно в случае вхождений свободных переменных, и в результате этой работы алгоритм приобрел такой вид. Современное, менее формальное описание алгоритма доступнее для понимания, что, например, упрощает реализацию алгоритма.